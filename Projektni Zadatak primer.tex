%%%%%%%%%%%%%%%%%%%%%%%%%%%%%%%%%%%%%%%%%
% LaTeX templejt za područje bivše Jugoslavije
% Version 1.0 (13/3/22)
%
%
% Autor: Prometej
%
%%%%%%%%%%%%%%%%%%%%%%%%%%%%%%%%%%%%%%%%%
%-----------------------------------------------------------------------
%	PAKETI
%-----------------------------------------------------------------------
\documentclass[12pt,serbian]{article} % Definisanje klase dokumenta i jezika -- srpski jezik
%\documentclass[12pt, croatian]{article}
% Definisanje klase dokumenta i jezika -- hrvatski jezik
\usepackage[T1,T2A]{fontenc}
\usepackage[utf8]{inputenc}
\usepackage[serbian]{babel}
%\usepackage[Croatian]{babel}
\usepackage[margin=0.8in,left=0.8in,includefoot]{geometry}
\usepackage{ragged2e}
\def\dj{d\kern-0.4em\char"16\kern-0.1em}
\def\Dj{\mbox{\raise0.3ex\hbox{-}\kern-0.4em D}}
\usepackage{graphicx}
\usepackage{amsmath}
\usepackage{float}
\usepackage{amsfonts}
\usepackage{setspace}
\usepackage{pdfpages}
\usepackage{titlesec}
\usepackage{tabularx}
\usepackage[shortlabels]{enumitem}
\usepackage{booktabs}
\usepackage{bm}
\usepackage{xcolor}
\usepackage{multicol}
\usepackage{csvsimple}
\usepackage[nottoc,numbib]{tocbibind}
\setcounter{tocdepth}{4} % dubina sadržaja 
% 1 section,
% 2 subsection,
% 3 subsubsection
% 4 paragraph
% dubina sadržaj
\setcounter{secnumdepth}{4}
\usepackage[hidelinks]{hyperref}
\numberwithin{figure}{section}
\numberwithin{table}{section}
\usepackage{hyperref}
\hypersetup{
    colorlinks=true,
    linkcolor=blue,
    filecolor=magenta,      
    urlcolor=cyan,
    citecolor={blue},
    pdftitle={Overleaf Example},
    }
%-----------------------------------------------------------------------
%	PAKETI
%-----------------------------------------------------------------------


%%%%%%%%%%%%%%%%%%% SAMO ZA UPOTREBU ĆIRILICE %%%%%%%%%%%%%%%%%%%%%%%%
% UKOLIKO SE KORISTI LATINICA ONDA OBRISATI ILI KOMENTIRATI OVAJ DEO %
%%%%%%%%%%%%%%%%%%%%%%%%%%%%%%%%%%%%%%%%%%%%%%%%%%%%%%%%%%%%%%%%%%%%%%
\AtBeginDocument{\renewcommand{\contentsname}{\LARGE{Садржај}}}
\AtBeginDocument{\renewcommand{\listfigurename}{Списак слика}}
\AtBeginDocument{\renewcommand{\figurename}{Слика}}
\AtBeginDocument{\renewcommand{\tablename}{Табела}}
\AtBeginDocument{\renewcommand{\listtablename}{Списак табела}}
\AtBeginDocument{\renewcommand{\glossaryname}{Списак кориштених ознака}}
\AtBeginDocument{\renewcommand{\abstractname}{Сажетак}}
%%%%%%%%%%%%%%%%%%%%%%%%%%%%%%%%%%%%%%%%%%%%%%%%%%%%%%%%%%%%%%%%%%%%%%
% UKOLIKO SE KORISTI LATINICA ONDA OBRISATI ILI KOMENTIRATI OVAJ DEO %
%%%%%%%%%%%%%%%%%%% SAMO ZA UPOTREBU ĆIRILICE %%%%%%%%%%%%%%%%%%%%%%%%
%____________________________________________________________________%
%____________________________________________________________________%
%____________________________________________________________________%
%____________________________________________________________________%
%____________________________________________________________________%
%____________________________________________________________________%
%%%%%%%%%%%%%%%%%%%%%%%%%%%%%%%%%%%%%%%%%%%%%%%%%%%%%%%%%%%%%%%%%%%%%%
%%%%%%%%%%%%%%&&&&&&&& POČETAK DOKUMENTA %%%%%%%%%%%%%%%%%%%%%%%%%%%%%
%%%%%%%%%%%%%%%%%%%%%%%%%%%%%%%%%%%%%%%%%%%%%%%%%%%%%%%%%%%%%%%%%%%%%%

\begin{document}
\onehalfspacing
\justifying
\sloppy
\begin{titlepage}
	\begin{center}
	\line(1,0){500}\\
	[0.5cm]
	\huge{\bfseries Назив предмета}\\ % Naziv predmeta i sl.
	[2mm]
	\line(1,0){500}\\
	\textsc{\large{Пројектни задатак}}\\% Tip projekta i sl.
	[1.5cm]	
	 \vspace*{2\baselineskip}
    \begin{center} % Centriranje slike, teksta...
\resizebox{0.3\textwidth}{!}{ 
\includegraphics{{example-image-a}} % Logo fakulteta ili slično
} %close resizebox    
\end{center}
    {\Large Машински факултет Универзитета у Београду} \\ {\large Катедра за ...} \\ % Ime ustanove i katedre
   
	\end{center}

	\begin{flushright}
	\vfill
	\textsc{\large Ime i prezime učesnika 1 \\ % imena autora
	Ime i prezime učesnika 2\\
	Ime i prezime učesnika 3\\}
	\end{flushright}
	
\end{titlepage}
\newpage
\tableofcontents
\newpage
\listoffigures
\newpage
\listoftables
\newpage

\section{Odeljak}

Zavisnost dinamičke viskoznosti i temperature fluida je prikazana u tabeli \ref{tab:din_vis}.
\begin{table}[H]
\small
\centering
\begin{tabular}{@{}lcc@{}}
\textbf{Fluid} & \textbf{Temperatura} [$^{\circ}$C] & \textbf{Dinamička viskoznost} [Pa s] \\
\bottomrule
Vazduh/Zrak  & 20  & $18.5 \cdot 10^{-6}$   \\

Vazduh/Zrak  & 100  & $21.74 \cdot 10^{-6}$   \\

Voda & 20 & $1.0016 \cdot 10^{-3}$\\

Voda & 100 &  $0.2822 \cdot 10^{-3}$\\

Med & 20 & 2000 $\div$ 10000\\

VW G13 (50--50) & 20  & $28.9 \cdot 10^{-3}$\\

Motorno ulje & 0 & 1.328\\

Motorno ulje & 100 & 0.01316\\
\bottomrule
\end{tabular}
\caption{Vrednosti dinamičke viskoznosti sveprisutnih fluida.}
\label{tab:din_vis}
\end{table}


\subsection{Pododeljak}
Između ostalog, Rejnoldsovo\footnote{Osborne Reynolds  (\textit{{1842-1912}}) je bio irski istaknuti naučnik iz oblasti mehanike fluida.} osrednjavanje Navije--Stoksovih jednačina. Prikaz drugog slova abecede je dat na slici \ref{fig:slovo_B}.


\begin{figure}[H]
    \centering
    \includegraphics[width=0.4\textwidth]{example-image-b}
    \caption{Slovo B.}
    \label{fig:slovo_B}
\end{figure}



\subsubsection{Podpododeljak}

\begin{equation}
    x:\ \displaystyle \rho \left(\frac{\partial \bar{u} }{\partial t}\right) + \rho \left(\bar{u}\frac{\partial \bar{u} }{\partial x}+ \bar{v}\frac{\partial \bar{u} }{\partial y}\right) = - \frac{\partial \bar{p} }{\partial x} + \mu \nabla^{2}\bar{u} - \rho \left(\overline{u^{\prime}\frac{\partial u^{\prime}}{\partial x}} +\overline{v^{\prime}\frac{\partial u^{\prime}}{\partial y}}\right)
\end{equation}
\begin{equation}
    y:\ \displaystyle \rho \left(\frac{\partial \bar{v} }{\partial t}\right) + \rho \left(\bar{u}\frac{\partial \bar{v} }{\partial x}+ \bar{v}\frac{\partial \bar{v} }{\partial y}\right) = - \frac{\partial \bar{p} }{\partial y} + \mu \nabla^{2}\bar{v} - \rho \left(\overline{u^{\prime}\frac{\partial v^{\prime}}{\partial x}} +\overline{v^{\prime}\frac{\partial v^{\prime}}{\partial y}}\right)
\end{equation}

\clearpage
\section{Odeljak}

\end{document}
