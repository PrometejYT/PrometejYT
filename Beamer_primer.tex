%%%%%%%%%%%%%%%%%%%%%%%%%%%%%%%%%%%%%%%%%
% Beamer Presentation templejt za područje bivše Jugoslavije
% Version 1.0 (13/3/22)
%
%
% Autor: Prometej
%
%%%%%%%%%%%%%%%%%%%%%%%%%%%%%%%%%%%%%%%%%

%-----------------------------------------------------------------------
%	PAKETI i TEME
%-----------------------------------------------------------------------

\documentclass{beamer} % Definisanje klase dokumenta
\usepackage[T1]{fontenc} 
\usepackage[T1,T2A]{fontenc}
\usepackage{graphicx} % Paket kojima se omogućuje prikaz slika
\usepackage{booktabs} % itd.
\usepackage{amsmath}
\usepackage{url,color}
\usepackage{subfigure}
\usepackage{amsthm,amsfonts,amssymb,amscd,amsxtra}
\usepackage{ragged2e}
\usepackage{wasysym}
\setbeamertemplate{caption}[numbered]
\usepackage{textcomp}
\renewcommand{\raggedright}{\justifying}
\usepackage[serbian]{babel}
\mode<presentation> {
\setbeamertemplate{section in toc}[sections numbered]
\setbeamertemplate{subsection in toc}[subsections numbered]
\usepackage{circuitikz}
\usetikzlibrary{patterns}
% Prilikom pravljenje dokumenata klase Beamer, postoje raznovrsne teme  % na raspolaganju korisnicima.
% Temama su definisane boje, okviri. 
% Odkomentirajte različite teme (brisanjem %) i vidite koja je          % najpodobnija.

%\usetheme{default}
%\usetheme{AnnArbor}
%\usetheme{Antibes}
%\usetheme{Bergen}
%\usetheme{Berkeley}
%\usetheme{Berlin}
%\usetheme{Boadilla}
%\usetheme{CambridgeUS}
%\usetheme{Copenhagen}
%\usetheme{Darmstadt}
%\usetheme{Dresden}
%\usetheme{Frankfurt}
%\usetheme{Goettingen}
%\usetheme{Hannover}
%\usetheme{Ilmenau}
%\usetheme{JuanLesPins}
%\usetheme{Luebeck}
\usetheme{Madrid}
%\usetheme{Malmoe}
%\usetheme{Marburg}
%\usetheme{Montpellier}
%\usetheme{PaloAlto}
%\usetheme{Pittsburgh}
%\usetheme{Rochester}
%\usetheme{Singapore}
%\usetheme{Szeged}
%\usetheme{Warsaw}


% Ispod liste za teme, se nalazi lista boja.

%\usecolortheme{albatross}
%\usecolortheme{beaver}
%\usecolortheme{beetle}
%\usecolortheme{crane}
%\usecolortheme{dolphin}
%\usecolortheme{dove}
%\usecolortheme{fly}
%\usecolortheme{lily}
%\usecolortheme{orchid}
%\usecolortheme{rose}
%\usecolortheme{seagull}
%\usecolortheme{seahorse}
\usecolortheme{whale}
%\usecolortheme{wolverine}


%\setbeamertemplate{navigation symbols}{} % Ako želite da uklonite oznake za navigaciju kroz dokument, koje se nalaze na stranicama odkomentirajte ovu liniju koda.


}

%%%%%%%%%%%%%%%%%%%%%%%%%%%%%%%%%%%%%%%%%%%%%%%%%%%%%%%%%%%%%%%%%%%
%%%%%%%%%%%%%%%%%%%%%%%%% NASLOVNA STRANA %%%%%%%%%%%%%%%%%%%%%%%%%
%%%%%%%%%%%%%%%%%%%%%%%%%%%%%%%%%%%%%%%%%%%%%%%%%%%%%%%%%%%%%%%%%%%

\title[Naslov]{Naslov 1} % Ovo je naslov koji će biti prikazan na dnu svake strane

\author{Autor} % Ime autora

\vspace*{-0.5cm}
\titlegraphic{\includegraphics[width=3cm]{example-image-a}\qquad} % LOGO odnosno slika

\date{\today} % Datum - može se promeniti tako da ispisuje datum osim današnjeg

\newcommand{\R}{\mathbb{R}}
\newcommand{\x}{\textbf{x}}
\newcommand{\y}{\textbf{y}}
\renewcommand{\qedsymbol}{$\blacksquare$}
\newcommand{\dom}{\mathrm{dom}}
\newcommand{\ad}{\mathrm{ad}}
\newcommand{\gerado}{\mathrm{span}}
\setbeamertemplate{section in toc}[sections numbered]
\setbeamertemplate{subsection in toc}[subsections numbered]

\begin{document}

\begin{frame}
\titlepage % Ispis naslova
\end{frame}

{
\begin{frame}
\frametitle{Sadržaj} % Sadržaj rada
\tableofcontents
\end{frame}
}


%%%%%%%%%%%%%%%%%%%%%%%%%%%%%%%%%%%%%%%%%%%%%%%%%%%%%%%%%%%%%%%%%%%
%%%%%%%%%%%%%%%%%%%%%%%%% POČETAK RADA %%%%%%%%%%%%%%%%%%%%%%%%%%%%
%%%%%%%%%%%%%%%%%%%%%%%%%%%%%%%%%%%%%%%%%%%%%%%%%%%%%%%%%%%%%%%%%%%

%-------------------------------------------------%
\section{Uvod} % Naslov koji se ispisuje u sadržaj
\frame{
\frametitle{\textbf{Uvod}} % Naslov koji se ispisuje na stranici

\setbeamertemplate{itemize items}[square] % Vrsta obeležavača spiska, uvom slučaju je to crni kvadrat
  \begin{itemize}
    \item Prvi element spiska
    \vspace{0.3cm} % Deklarisanje odstojanja između elemenata spiska
    \item Drugi element spiska
    \vspace{0.3cm}
    \item Treći element spiska
    \vspace{0.3cm}
    \item Може и ћирилицом
  \end{itemize}
}

%-------------------------------------------------%
\begin{frame}
\section{Glava 1}
\frametitle{\textbf{Naslov Glave 1}}


\end{frame}

%-------------------------------------------------%
\section{Glava 2}
\begin{frame}
    \frametitle{\textbf{Naslov Glave 2}}


\end{frame}
%-----------------------------------------------%
\subsection{Podnaslov}
\begin{frame}
    \frametitle{Naslov podnaslova 3.1.}

\begin{figure} % Upuststvo za uporedi prikaz dveju slika
   \begin{minipage}[b]{0.45\linewidth}
            \centering
            \includegraphics[width=\textwidth]{example-image-b} 
            \caption{Opis prve slike.}
            \label{fig:prva_slika} % Označavanje slike, koristi se za potonje pozivanje na sliku u daljem tekstu
        \end{minipage}
        \hspace{0.5cm}
        \begin{minipage}[b]{0.45\linewidth}
            \centering
            \includegraphics[width=\textwidth]{example-image-a} 
            \caption{Opis druge slike.}
            \label{fig:druga_slika}
    \end{minipage}
\end{figure}  

\begin{itemize}
    \item Na slici \ref{fig:druga_slika} možete videti 1. slovo abecede
\end{itemize}
  
\end{frame}
%------------------------------------------------%
\begin{frame}
\frametitle{Jednačine} %
\setbeamertemplate{itemize items}[square]
 \begin{itemize}
     \item Ispivanje jednačina je identično kao u drugim klasama \LaTeX--a 
     \vspace{0.1cm}
 \end{itemize}
 \begin{equation}
     \sigma = \varepsilon \cdot E
 \end{equation}

 
\end{frame}
%----------------------------------------------------%
\end{document}

